\SetPicSubDir{ch-Rice}
\SetExpSubDir{ch-Rice}

\chapter{Like White on Rice}
\label{ch:rice}
\vspace{2em}

\lipsum[1-4]

\section{Preliminaries}

\lipsum[5-8]

\begin{figure}[!t]
  \centering
  \includegraphics[width=.5\linewidth]{\Pic{jpg}{rice}}
  \vspace{\BeforeCaptionVSpace}
  \caption{A bowl of rice.}
  \label{Rice:fig:bowl_of_rice}
\end{figure}

\section{Methodology}

\autoref{Rice:fig:bowl_of_rice} shows a bowl of rice. 
\autoref{Rice:algo:sample} demonstrates the formatting of pseudo code. 
Please carefully check the source files and learn how to use this style. 
Importantly:

\begin{itemize}
\item Always state your input.

\item State the output if any. 

\item Always number your lines for quick referral.

\item Always declare and initialize your local variables. 

\item Always use \textbackslash gets for assignments.
\end{itemize}

\begin{algorithm}[!t]
\AlgoFontSize
\DontPrintSemicolon

\KwGlobal{max. calories of daily intake $\mathcal{C}$}
\KwGlobal{calories per bowl of rice $\mathcal{B}$}
\BlankLine

\KwIn{number of bowls of rice $n$}
\KwOut{calories intake}
\Proc{\fEatRice{$n$}}{
  $cal \gets n \times \mathcal{B} - \fDoExercise{n}$\;
  \uIf{$cal \geq \mathcal{C}$}{
    \Return $\mathcal{C}$\;
  }
  \Else{
    \Return $cal$\;
  }
}

\BlankLine

\KwIn{time duration (in minutes) of exercise $t$}
\KwOut{calories consumed}
\Func{\fDoExercise{$t$}}{
  $cal \gets 0$\;
  \lFor{$i \gets 1 \To t$}{$cal \gets cal + i$\;}
  \Return $cal$\;
}

\caption{Sample pseudo code of a dummy algorithm.}
\label{Rice:algo:sample}
\end{algorithm}

\section{Evaluation}

\lipsum[16-20]

\section{Summary}

\lipsum[21]
